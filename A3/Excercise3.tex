\documentclass[12pt,a4paper]{article}
\usepackage{amsfonts}
\usepackage{amsmath,amscd,amsbsy,amssymb,latexsym,url,bm,amsthm}
\usepackage{epsfig,graphicx,subfigure}
\usepackage{enumitem,balance}
\usepackage{wrapfig}
\usepackage{mathrsfs,euscript}
\usepackage[usenames]{xcolor}
\usepackage{hyperref}
\usepackage[vlined,ruled,commentsnumbered,linesnumbered]{algorithm2e}

%note: use \setcounter{exercise}{5} to change the numbering
\theoremstyle{definition}
\newtheorem{exercise}{Exercise}
\newtheorem*{solution}{Solution}

%copied from VE281. I don't know what they are used for. Some are used to expand the margin.
\setlength{\oddsidemargin}{-0.365in}
\setlength{\evensidemargin}{-0.365in}
\setlength{\topmargin}{-0.3in}
\setlength{\headheight}{0in}
\setlength{\headsep}{0in}
\setlength{\textheight}{10.1in}
\setlength{\textwidth}{7in}
\makeatletter \renewenvironment{proof}[1][Proof] {\par\pushQED{\qed}\normalfont\topsep6\p@\@plus6\p@\relax\trivlist\item[\hskip\labelsep\bfseries#1\@addpunct{.}]\ignorespaces}{\popQED\endtrivlist\@endpefalse} \makeatother
\makeatletter
\renewenvironment{solution}[1][Solution] {\par\pushQED{\qed}\normalfont\topsep6\p@\@plus6\p@\relax\trivlist\item[\hskip\labelsep\bfseries#1\@addpunct{.}]\ignorespaces}{\popQED\endtrivlist\@endpefalse} \makeatother

\begin{document}
\title{VE401 Assignment 3}
\author{Yang, Tiancheng 517370910259\\Qiu, Yuqing 518370910026\\Chang, Siyao 517370910024}

\maketitle

\newpage

\setcounter{exercise}{10}
\begin{exercise}
Bivariate Normal Distribution as a Mixture of Independent Normal Distributions
\end{exercise}
\begin{enumerate}[label=\roman*)]
    \item 1
    \item 2
    \item 3
    \item 5
    \item \begin{solution}
        We know the m.g.f. of normal distribution is
        \begin{equation*}
            m_x\left(t\right)=exp\left(u\times t-0.5t^2\times D\right)
        \end{equation*}
        where D is the variance. We send it in matrix terms.

        Let $X=(x1,x2)$. $M_x(t)=exp(u^Tt-0.5t^T\cdot D\cdot t)$, D is 
        \begin{equation*}
            \begin{pmatrix}
                \sigma_1^2&,0\\
                0&,\sigma_2^2
            \end{pmatrix}
        \end{equation*}
        , A is the transform matrix for Y=AX, u is matrix of expression.
        
        We already know X follows bi-normal distribution for 
        \begin{equation*}
            \begin{split}
                f_{x_1x_2}(X_1,X_2)&=P(X_1=x1,X_2=x_2)\\
                &=P(X_1=x_1)\cdot P(X_2=x_2)\\
                &=f_{x_1}(x)\cdot f_{x_2}(x)\\
                &=\sqrt{A}\cdot \int e^{(n^2)^B}dx_1\cdot \sqrt{A}\cdot \int e^{(m^2)^B}dx_2\\
                &=A\cdot \int e^{(m^2+n^2)^B} dx_1 dx_2
            \end{split}
        \end{equation*}
        It’s Bi-normal distribution’s pdf when $\rho=0$.Since $X_1$,$X_2$ are independent, $\rho=0$. So X follows Bi-normal distribution. Y=AX
        
        In matrix term,
        \begin{equation*}
            \begin{split}
                m_Y(t)&=E(e^Yt^T)=E(e^AXt^T)=E(e^{(A^Tt)^T}X)\\
                &=exp(u^TA^Tt-0.5(A^Tt)^T\cdot D\cdot A^Tt)\\
                &=exp((Au)^Tt-0.5(A^Tt)^T\cdot D\cdot A^Tt)\\
                &=exp((Au)^Tt-0.5(t)^T\cdot (A\cdot D\cdot A^T)\cdot t)\\
                &=exp((u')^Tt-0.5(t)^T\cdot (D')\cdot t)
            \end{split}
        \end{equation*}
        
        Obviously Y follows the same kind of distribution with only different parameters. Y follows bi-normal distribution.
    
        According to 3.10.3, 
        \begin{equation*}
            f_{x_1x_2}(x_1,x_2)=A\cdot exp(-0.5 < x-ux, \sum \ ^{-1} x(x-ux) > )
        \end{equation*}
        Hence
        \begin{equation*}
            f_{y_1y_2}(y_1,y_2)=A\cdot exp(-0.5<y-uy,\sum \ ^{-1}y(y-uy)>)
        \end{equation*}
        And $fy(y)=Ae^(m^2-2\rho mn+n^2)^B$
        
        with:

        $A=1/2p\sigma_1\sigma_2 \sqrt{1-\rho_2}$

        $B=-0.5/(1-\rho_2)$

        $m=(y_2-u_2)/\sigma_2$

        $n=(y_1-\mu_1)/\sigma_1$
    \end{solution}
\end{enumerate}
\end{document}

