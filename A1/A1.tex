\documentclass[12pt,a4paper]{article}
\usepackage{amsmath,amscd,amsbsy,amssymb,latexsym,url,bm,amsthm}
\usepackage{epsfig,graphicx,subfigure}
\usepackage{enumitem,balance}
\usepackage{wrapfig}
\usepackage{mathrsfs,euscript}
\usepackage[usenames]{xcolor}
\usepackage{hyperref}
\usepackage[vlined,ruled,commentsnumbered,linesnumbered]{algorithm2e}

%note: use \setcounter{exercise}{5} to change the numbering
\theoremstyle{definition}
\newtheorem{exercise}{Exercise}
\newtheorem*{solution}{Solution}

%copied from VE281. I don't know what they are used for. Some are used to expand the margin.
\setlength{\oddsidemargin}{-0.365in}
\setlength{\evensidemargin}{-0.365in}
\setlength{\topmargin}{-0.3in}
\setlength{\headheight}{0in}
\setlength{\headsep}{0in}
\setlength{\textheight}{10.1in}
\setlength{\textwidth}{7in}
\makeatletter \renewenvironment{proof}[1][Proof] {\par\pushQED{\qed}\normalfont\topsep6\p@\@plus6\p@\relax\trivlist\item[\hskip\labelsep\bfseries#1\@addpunct{.}]\ignorespaces}{\popQED\endtrivlist\@endpefalse} \makeatother
\makeatletter
\renewenvironment{solution}[1][Solution] {\par\pushQED{\qed}\normalfont\topsep6\p@\@plus6\p@\relax\trivlist\item[\hskip\labelsep\bfseries#1\@addpunct{.}]\ignorespaces}{\popQED\endtrivlist\@endpefalse} \makeatother

\begin{document}
\title{VE401 Assignment}
\author{Yang Tiancheng 517370910259}
\maketitle

\newpage

\begin{exercise}
Elementary Probability
\begin{solution}
We use Cardano's principle to get the probability. The number of ways to pick 120 people from 2000 individuals is
$$n_1=\frac{2000!}{120!\times(2000-120)!}$$
The number of ways that me and my friend are both chosen is equal to the number of ways to choose 118 people from 1998 individuals, which is
$$n_2=\frac{1998!}{118!\times(1998-118)!}$$
Therefore the probability that me and my friend will both be chosen is
$$\frac{n_2}{n_1}=\frac{\frac{1998!}{118!\times(1998-118)!}}{\frac{2000!}{120!\times(2000-120)!}}=0.357\%$$
\end{solution}
\end{exercise}

\begin{exercise}
Some Routine Calculations
\begin{enumerate}[label=\roman*)]
\item
\begin{proof}
    Since $A\subset B$, $B=A+B\backslash A$. Note that $A \cap B \backslash A = \emptyset $. Thus $P[B]=P[A]+P[B\backslash A]\geq P[A]$. Therefore $P[A]\leq P[B]$.
\end{proof}
\item
\begin{proof}
    Since A and B are independent, we have $P[A\cap B]=P[A]P[B]$. We know that $P[A]P[B]>0$ so $P[A\cap B]>0$. Thus $P[A\cap B]=\frac{|A\cap B|}{|S|}>0$, which means that $|A\cap B|> 0$. Therefore, $A\cap B\neq \emptyset$ and hence they are not mutually exclusive.
\end{proof}
\item
\begin{proof}
    First we have two trivial equations:
    $$P[A]=P[A\backslash (A\cap B)]+P[A\cap B]$$
    $$P[B]=P[B\backslash (A\cap B)]+P[A\cap B]$$
    We also have
    $$P[A\cup B]=P[A\backslash (A\cap B)]+P[B\backslash (A\cap B)]+P[A\cap B]$$.
    Therefore,
    $$P[A\cup B]=P[A]-P[A\cap B]+P[B]-P[A\cap B]+P[A\cap B]=P[A]+P[B]-P[A\cap B]$$
\end{proof}
\end{enumerate}
\end{exercise}
\begin{exercise}
    D'Alembert's Coins
    \begin{enumerate}[label=\roman*)]
        \item
        \begin{solution}
            No, it is not possible. If the coin is fair, we know that $$P[two\ heads]=P[no \ heads] < P[one head]$$. Now if the coin is biased, for that coin $P[head]\neq P[no head]$. And hence if it is tossed twice, $$P[two \ heads]\neq P[no \ heads]$$. Therefore, even though the coin can be biased, the three outcomes cannot have the same probability.
        \end{solution}
        \item 
        \begin{solution}
            No, it is not possible as well. Denote one coin as $A$, with $P[A,head]=a$. We know that $P[A,head]+P[A,tail]=1$ so $P[A,tail]=1-a$. Similarly, we also have coin $B$ with $b$ and $1-b$. Now, if the three outcomes have same probability, then $$a*b=(1-a)*(1-b)=1/3$$. From here we get that $a+b=1$. Now we calculate $$a*b=a*(1-a)=\frac{1}{3}$$ which has no real solution. Therefore, it is impossible to make the coins so that D'Alembert's claim is true.
        \end{solution}
    \end{enumerate}
\end{exercise}
\begin{exercise}
    Independence
    \begin{enumerate}[label=\roman*)]
        \item
        \begin{solution}
            Denotion of events:
            \newline P1: a participant from the first group is chosen;
            \newline P2: a participant from the second group is chosen;
            \newline A: the participant replies "yes" to the second question;
            \newline Now we list the known probabilities: $P[P1]=50\%$, $P[P2]=50\%$, $P[A|P1]=17\%$, $P[A|P2]=3\%$. We want to know the total probability of $P[A]$.
            \begin{equation*}
                P[A]=P[A|P1]*P[P1]+P[A|P2]*P[P2]=17\% * 50\% + 3\% * 50\% = 10\%
            \end{equation*}
            Therefore, this probability is $10\%$.
        \end{solution}
        \item 
        \begin{solution}
            No. We have $P[A|P1] > P[A]$. Therefore it is not independent.
        \end{solution}
    \end{enumerate}
\end{exercise}
\begin{exercise}
    This one may need a little thinking about... Though it doesn't.
    \begin{solution}
        We denote the event that a chip is defective by $D$. The event that a chip is stolen is $S$. $\neg S$ means that a chip is not stolen. Thieves stole the chips before inspection, so $P[D|S]=50\% $. We know that $P[S]=1\%$ of the chips is illegal, hence $P[\neg S] = 99\%$ of the chips is legally marketed. For those chips that are legally marketed, their probability of being defective is $5\%$, because of the inspection. This means that if the chips is not stolen, the probability of being defective is $P[D|\neg S]=5\%$. Now we apply the Bayes' theorem to get the probability we want to calculate, i.e., $P[S|D]$.
        \begin{equation*}
            P[S|D]=\frac{P[D|S]P[S]}{P[D|S]P[S]+P[D|\neg S]P[\neg S]}=\frac{0.5\times 0.01}{0.5\times 0.01+0.05\times 0.99}=9.17\%
        \end{equation*}
    \end{solution}
\end{exercise}
\begin{exercise}
    Mounty Hall in Prison?
    \begin{solution}
        They are both wrong. Conclusions first: the prisoner is wrong because the warden would have given him some information; the warden is wrong because the one who have a bigger chance of dying would not have been prisoner $A$.
        \newline Suppose that the warden tells $A$ that prisoner $B$ is not to be executed. We denote this event by $B^*$. The event that prisoner $X$ is to be executed is denoted by $DX$. By Bayes' Theorem, we have
        \begin{equation*}
            P[DA|B^*]=\frac{P[B^*|DA]P[DA]}{P[B^*|DA]P[DA]+P[B^*|DB]P[DB]+P[B^*|DC]P[DC]}=\frac{\frac{1}{2}\times \frac{1}{3}}{\frac{1}{2}\times \frac{1}{3}+0+1\times \frac{1}{3}}=\frac{1}{3}
        \end{equation*}
        However, for prisoner $C$, the probability that he is to be executed is then
        \begin{equation*}
            P[DC|B^*]=\frac{P[B^*|DC]P[DC]}{P[B^*|DC]P[DC]+P[B^*|DB]P[DB]+P[B^*|DA]P[DA]}=\frac{1\times \frac{1}{3}}{1\times \frac{1}{3}+0+\frac{1}{2}\times \frac{1}{3}}=\frac{2}{3}
        \end{equation*}
        Therefore, neither of them is right.
    \end{solution}
\end{exercise}
\begin{exercise}
   Two Children Paradox - Birthday Party\\
   \begin{solution}
Denote the events as:\\
A: The lady's other child is a girl;\\
B: The lady's boy is born in July;\\
Then, according to the problem, what we need to calculate is $P[A|B]$.\\
We define the sample space of the gender as $\{(b,b),(b,g),(g,b)\}$, where 'b' means the child is a boy and 'g' means the child is a girl and the former child is older than the latter. Each sample point has the same probability, i.e. $$P[(b,b)]=P[(b,g)]=P[(g,b)]=\frac{1}{3}$$
For $(b,g)$ and $(g,b)$, the probability of the boy born in July should be $\frac{1}{12}$.\\
For $(b,b)$, the probability of at least one boy born in July can be calculated as follows:
\begin{enumerate}[label=\roman*)]
    \item Only one boy is born in July: $\frac{1}{12}\times\frac{11}{12}\times 2=\frac{22}{144}$
    \item Two boys are both born in July: $\frac{1}{12}\times\frac{1}{12}=\frac{1}{144}$
\end{enumerate}
Then, the probability of at least one boy born in July is $\frac{22}{144}+\frac{1}{144}=\frac{23}{144}$
   \end{solution}

Then, using the formula of conditional formula,
\begin{align*}
    P[A|B]&=\frac{P[A\cap B]}{P[B]}\\
    &=\frac{\frac{1}{12}\times \frac{1}{3}+\frac{1}{12}\times\frac{1}{3}}{\frac{1}{12}\times\frac{1}{3}+\frac{1}{12}\times\frac{1}{3}+\frac{23}{144}\times \frac{1}{3}}\\
    &=\frac{24}{47}
\end{align*}
\end{exercise}
\end{document}
