\documentclass[12pt,a4paper]{article}
\usepackage{amsmath,amscd,amsbsy,amssymb,latexsym,url,bm,amsthm}
\usepackage{epsfig,graphicx,subfigure}
\usepackage{enumitem,balance}
\usepackage{wrapfig}
\usepackage{mathrsfs,euscript}
\usepackage[usenames]{xcolor}
\usepackage{hyperref}
\usepackage[vlined,ruled,commentsnumbered,linesnumbered]{algorithm2e}

%note: use \setcounter{exercise}{5} to change the numbering
\theoremstyle{definition}
\newtheorem{exercise}{Exercise}
\newtheorem*{solution}{Solution}

%copied from VE281. I don't know what they are used for. Some are used to expand the margin.
\setlength{\oddsidemargin}{-0.365in}
\setlength{\evensidemargin}{-0.365in}
\setlength{\topmargin}{-0.3in}
\setlength{\headheight}{0in}
\setlength{\headsep}{0in}
\setlength{\textheight}{10.1in}
\setlength{\textwidth}{7in}
\makeatletter \renewenvironment{proof}[1][Proof] {\par\pushQED{\qed}\normalfont\topsep6\p@\@plus6\p@\relax\trivlist\item[\hskip\labelsep\bfseries#1\@addpunct{.}]\ignorespaces}{\popQED\endtrivlist\@endpefalse} \makeatother
\makeatletter
\renewenvironment{solution}[1][Solution] {\par\pushQED{\qed}\normalfont\topsep6\p@\@plus6\p@\relax\trivlist\item[\hskip\labelsep\bfseries#1\@addpunct{.}]\ignorespaces}{\popQED\endtrivlist\@endpefalse} \makeatother

\begin{document}
\title{VE401 Assignment}
\author{Yang Tiancheng 517370910259}
\maketitle

\newpage

\begin{exercise}
Elementary Probability
\begin{solution}
We use Cardano's principle to get the probability. The number of ways to pick 120 people from 2000 individuals is
$$n_1=\frac{2000!}{120!\times(2000-120)!}$$
The number of ways that me and my friend are both chosen is equal to the number of ways to choose 118 people from 1998 individuals, which is
$$n_2=\frac{1998!}{118!\times(1998-118)!}$$
Therefore the probability that me and my friend will both be chosen is
$$\frac{n_2}{n_1}=\frac{\frac{1998!}{118!\times(1998-118)!}}{\frac{2000!}{120!\times(2000-120)!}}=0.357\%$$
\end{solution}
\end{exercise}

\begin{exercise}
Some Routine Calculations
\begin{enumerate}[label=\roman*)]
\item
\begin{proof}
    Since $A\subset B$, $B=A+B\backslash A$. Note that $A \cap B \backslash A = \emptyset $. Thus $P[B]=P[A]+P[B\backslash A]\geq P[A]$. Therefore $P[A]\leq P[B]$.
\end{proof}
\item
\begin{proof}
    Since A and B are independent, we have $P[A\cap B]=P[A]P[B]$. We know that $P[A]P[B]>0$ so $P[A\cap B]>0$. Thus $P[A\cap B]=\frac{|A\cap B|}{|S|}>0$, which means that $|A\cap B|> 0$. Therefore, $A\cap B\neq \emptyset$ and hence they are not mutually exclusive.
\end{proof}
\item
\begin{proof}
\end{proof}
\end{enumerate}
\end{exercise}

 \end{document}
